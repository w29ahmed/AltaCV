%%%%%%%%%%%%%%%%%
% This is an sample CV template created using altacv.cls
% (v1.1.3, 30 April 2017) written by LianTze Lim (liantze@gmail.com). Now compiles with pdfLaTeX, XeLaTeX and LuaLaTeX.
% 
%% It may be distributed and/or modified under the
%% conditions of the LaTeX Project Public License, either version 1.3
%% of this license or (at your option) any later version.
%% The latest version of this license is in
%%    http://www.latex-project.org/lppl.txt
%% and version 1.3 or later is part of all distributions of LaTeX
%% version 2003/12/01 or later.
%%%%%%%%%%%%%%%%

%% If you need to pass whatever options to xcolor
\PassOptionsToPackage{dvipsnames}{xcolor}

%% If you are using \orcid or academicons
%% icons, make sure you have the academicons 
%% option here, and compile with XeLaTeX
%% or LuaLaTeX.
% \documentclass[10pt,a4paper,academicons]{altacv}

%% Use the "normalphoto" option if you want a normal photo instead of cropped to a circle
% \documentclass[10pt,a4paper,normalphoto]{altacv}

\documentclass[10pt,a4paper]{altacv}

%% AltaCV uses the fontawesome and academicon fonts
%% and packages. 
%% See texdoc.net/pkg/fontawecome and http://texdoc.net/pkg/academicons for full list of symbols.
%% 
%% Compile with LuaLaTeX for best results. If you
%% want to use XeLaTeX, you may need to install
%% Academicons.ttf in your operating system's font 
%% folder.


% Change the page layout if you need to
\geometry{left=1cm,right=9cm,marginparwidth=6.8cm,marginparsep=1.2cm,top=1.25cm,bottom=1.25cm,footskip=2\baselineskip}

% Change the font if you want to.

% If using pdflatex:
\usepackage[utf8]{inputenc}
\usepackage[T1]{fontenc}
\usepackage[default]{lato}

\usepackage[hidelinks]{hyperref}
\usepackage{textcomp}

% If using xelatex or lualatex:
% \setmainfont{Lato}

% Change the colours if you want to
\definecolor{Mulberry}{HTML}{72243D}
\definecolor{DarkRed}{HTML}{962938}
\definecolor{SlateGrey}{HTML}{2E2E2E}
\definecolor{LightGrey}{HTML}{666666}
\colorlet{heading}{DarkRed}
\colorlet{accent}{Mulberry}
\colorlet{emphasis}{SlateGrey}
\colorlet{body}{Black}

% Change the bullets for itemize and rating marker
% for \cvskill if you want to
\renewcommand{\itemmarker}{{\small\textbullet}}
\renewcommand{\ratingmarker}{\faCircle}

%% sample.bib contains your publications
\addbibresource{sample.bib}

% Commands to include a company logo for each /cvevent
\newcommand{\companylogo}[2]
  {\hfill\raisebox{-2\baselineskip}[0pt][0pt]%
    {\href{#2}{\includegraphics[height=3\baselineskip]{#1}}}}
    
\newcommand{\projectlogo}[2]
  {\hfill\raisebox{-1\baselineskip}[0pt][0pt]%
    {\href{#2}{\includegraphics[height=2\baselineskip]{#1}}}}

\begin{document}
\name{WALEED \textcolor{DarkRed}{AHMED}}
% \tagline{1B Computer Engineering}
% \photo{2.8cm}{Globe_High}
\personalinfo{%
  % Not all of these are required!
  % You can add your own with \printinfo{symbol}{detail}
  \email{w29ahmed@edu.uwaterloo.ca}
  \phone{647-708-7272}
  % \mailaddress{Address, Street, 00000 County}
  \linkedin{\href{https://www.linkedin.com/in/waleed-a/}{linkedin.com/in/waleed-a}}
  \github{\href{https://github.com/w29ahmed}{github.com/w29ahmed}}
  %% You MUST add the academicons option to \documentclass, then compile with LuaLaTeX or XeLaTeX, if you want to use \orcid or other academicons commands.
%   \orcid{orcid.org/0000-0000-0000-0000}
}

%% Make the header extend all the way to the right, if you want. 
\begin{fullwidth}
\makecvheader
\end{fullwidth}

\cvsection[page1sidebar]{Skills}
\smallskip
\cvtag{C++}
\cvtag{C\#}
\cvtag{Python}
\cvtag{JavaScript}
\cvtag{HTML/CSS}
\cvtag{Java}
\cvtag{MATLAB}

\divider

\cvtag{OpenCV}
\cvtag{ROS}
\cvtag{.NET}
\cvtag{Git}
\cvtag{Bash}
\cvtag{Linux}
\cvtag{Arduino}
\cvtag{VHDL}

%% Provide the file name containing the sidebar contents as an optional parameter to \cvsection.
%% You can always just use \marginpar{...} if you do
%% not need to align the top of the contents to any
%% \cvsection title in the "main" bar.
\cvsection{Experience}

\cvevent{Software Engineering Intern \companylogo{pictures/Synaptive_Logo.jpg}{https://www.synaptivemedical.com/}}{Synaptive Medical}{Sep 2018 -- Present}{Toronto, ON}
\begin{itemize}
\item Implemented a colour contrast enhancement algorithm in \textbf{C\#} using three dimensional look up tables in order to improve visibility of biological tissue during surgical procedures
\item Created a front-end interface using \textbf{JavaScript}, \textbf{HTML}, and \textbf{CSS} for testing/diagnostics of colour correction algorithms through an in-house web API paired with a \textbf{foreign function interface (FFI)} between \textbf{Visual C++} and \textbf{C\#}
\item Refactored various \textbf{dynamically linked libraries} for colour corrections into a single library for more efficient and organized software architecture
\end{itemize}
\divider

\cvevent{Industrial Imaging Software Developer \companylogo{pictures/PPO_Logo.png}{https://ppo.ca/}}{ P\&P Optica}{Jan 2018 -- Apr 2018}{Waterloo, ON}
\begin{itemize}
\item Developed software for industrial imaging applications on \textbf{Linux} machines with \textbf{Git} version control in an \textbf{Agile} environment 
\item Implemented image correction algorithms and post-processing for industrial cameras in \textbf{Python} using \textbf{Numpy}, \textbf{OpenCV}, and \textbf{Matplotlib}
\item Refactored data handling modules for efficient file input/output and wrote unit tests for them in \textbf{Python} using \textbf{Pytest}
\item Refactored camera control modules in \textbf{C/C++} that use the \textbf{Camera Link} serial protocol to interface with the camera for control purposes
\item Documented software design decisions and a troubleshooting guide to efficiently debug issues pertaining to image acquisition
\end{itemize}

\cvsection{Activities}

\cvevent{Software Team Member \companylogo{pictures/UW_Robotics.png}{http://uwrobotics.uwaterloo.ca/}}{UW Robotics}{Apr 2018 -- Current}{}
\begin{itemize}
\item Developed software architecture using a \textbf{Linux} based framework: \textbf{ROS} (Robot Operating System), for efficient package management and communication between \textbf{machine vision} modules in \textbf{C++}
\item Used \textbf{CUDA}, NVIDEA's parallel computing platform for GPU optimization of \textbf{OpenCV} code in \textbf{C++} for lane and object detection
\end{itemize}

\divider

\cvevent{Engineering Laboratories \& Projects \companylogo{pictures/ECE_Logo.png}{https://uwaterloo.ca/electrical-computer-engineering/}}{University of Waterloo}{Apr 2018 -- Current}{}
\begin{itemize}
\item ECE 108: Analyzed common casino games with set logic/probability and modeled a social network using a relational database in \textbf{C++}
\item ECE 124: Programmed FPGAs in \textbf{VHDL} using digital logic
\end{itemize}

\end{document}
