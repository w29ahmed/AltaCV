%% Yeah I didn't spend too much time making all the 
%% spacing consistent... sorry. Use \smallskip, \medskip, 
%% \bigskip, \vpsace etc to make ajustments.

\cvsection{Design Teams}

\cvevent{Path Planning Core Member \companylogo{pictures/Wato.png}{}{https://watonomous.ca}}{WATonomous}{Sep 2019 -- Dec 2019}{}
\begin{itemize}
% \item Developing a simulation tool to efficiently test trajectory planning and costmap generation  using \textbf{MATLAB}, \textbf{C++} and \textbf{ROS}
\item Simulation software development for a level 4 autonomous vehicle competing in the \href{https://www.sae.org/attend/student-events/autodrive-challenge/}{\textbf{SAE AutoDrive Challenge}}
\end{itemize}
\divider

\cveventwithgit{Software Team Lead \companylogo{pictures/UW_Robotics.png}{}{http://uwrobotics.uwaterloo.ca/}}{UW Robotics}{Apr 2018 -- Aug 2019}{}{\href{https://github.com/uwrobotics/RR2019-Hummingbot-Software}{github.com/uwrobotics/RR2019}}
\begin{itemize}
\item Managed development for a robot that competed in the \href{https://iarrc.org/}{\textbf{International Autonomous Robot Racing Competition}} 
\item Developed software architecture in \textbf{ROS} and \textbf{C++} for \textbf{perception}, \textbf{mapping}, and \textbf{path planning} using a stereo camera, IMU, and LiDAR sensor
% \item Implemented a lane detection algorithm capable of handling variable lane widths, curvature, and lighting conditions at a maximum of 25 Hz using \textbf{OpenCV} 
% \item Introduced a new lightweight traffic light detection algorithm using \textbf{OpenCV} that reduced overhead and false positive rate
\end{itemize}

\cvsection{Projects}

\cveventwithgit{Synviz \companylogo{pictures/synviz.png}{}{https://devpost.com/software/synviz}}{\href{https://uofthacks.com/}{UofTHacks VII (3rd place)}}{Jan 2020}{}{\href{https://github.com/w29ahmed/Synviz}{github.com/w29ahmed/Synviz}}
\begin{itemize}
\item Built an IoT device that could decode spoken text from facial input
\item Developed a backend web server in Python using \textbf{Flask}, \textbf{Google Cloud Storage}, \textbf{OpenCV}, and \textbf{TensorFlow}
\end{itemize}
\divider

\cveventwithgit{Agilite \companylogo{pictures/Agilite_Logo.png}{}{https://devpost.com/software/agilite}}{\href{https://www.deltahacks.com/}{DeltaHacks V}}{Jan 2019}{}{\href{https://github.com/w29ahmed/Agilite}{github.com/w29ahmed/Agilite}}
\begin{itemize}
\item Built a \textbf{Python} backend using \textbf{OpenCV} and \textbf{TensorFlow} capable of recognizing handwritten text from an agile board
% \item -	Interfaced the output of the python processing pipeline through Json to PostgreSQL, which was used as the database management system for the web application 
\end{itemize}

% \cvproject{CropIT \href{https://hackwestern.com/}{(Hack Western 5)}} {\href{https://github.com/w29ahmed/CropIT-Android}{github.com/w29ahmed/CropIT-Android}}
% \begin{itemize}
% \item Android app that classifies agricultural crops as healthy or unhealthy built using \textbf{Java} and \textbf{Android Studio}
% \item Trained a \textbf{convolutional neural network (CNN)} in \textbf{Python} using \textbf{TensorFlow/Keras }
% \end{itemize}

% \divider

% \cvproject{Toronto Raptors Image Classifier}
% {\href{https://github.com/w29ahmed/toronto-raptors-classifier}{github.com/w29ahmed/toronto-raptors-classifier}}
% \begin{itemize}
% \item Utilized transfer learning on Google's Inception v3 image classifier to identify basketball players using \textbf{TensorFlow}
% \end{itemize}

% \divider

% \cvproject{Arduino Jukebox}{\href{https://github.com/w29ahmed/Arduino-JukeBox}{github.com/w29ahmed/Arduino-JukeBox}}
% \begin{itemize}
% \item Programmed in \textbf{C++} to cycle through and play songs displayed on a LCD screen
% \item Songs are hard coded frequency patterns digitally sent to a piezoelectric speaker
% \end{itemize}

% \divider

% \cvproject{Android Notes App}{\href{https://github.com/w29ahmed/Notes_App}{github.com/w29ahmed/Notes\_App}}
% \begin{itemize}
% \item Simple but efficient note taking app for Android API levels 15 and above constructed using \textbf{Java}, \textbf{XML}, and a \textbf{SQLite Database }
% \end{itemize}

\cvsection{Education}

% \cvevent{B.ASc Computer Engineering
\cvevent{B.ASc Computer Engineering, Artificial Intelligence Option
\uwlogo{pictures/UW.jpg}{}{https://uwaterloo.ca/future-students/programs/computer-engineering}}{University of Waterloo}{Sep 2017 - Apr 2022}{}

% \divider

% {\large\color{emphasis}Online Coursework\par}
% \smallskip
\begin{itemize}
% \item \href{https://www.coursera.org/account/accomplishments/records/SPDNXPPF76PH}{Coursera: Visual Perception for Self-Driving Cars}
% \item \href{https://www.coursera.org/account/accomplishments/records/GA2AVP9VE9KJ}{Coursera: State Estimation and Localization for Self-Driving Cars}
% \item \href{https://www.coursera.org/account/accomplishments/records/HPCUJFQWZJHB}{Coursera: Intro to Self-Driving Cars}
\item \href{https://www.coursera.org/specializations/self-driving-cars}{Coursera: Self-Driving Cars Specialization}
\item \href{https://www.coursera.org/account/accomplishments/records/AELG6KXXJ88V}{Coursera: Machine Learning (Andrew Ng)}
\item \href{https://www.udemy.com/certificate/UC-TJRJ90AG/}{Udemy: Computer Vision}
% \item \href{https://www.udemy.com/machinelearning}{Udemy: Machine Learning A-Z}
\end{itemize}

% \cvsection{Interests}
% \cvtag{Autonomous Vehicles}
% \cvtag{Deep Learning}
% \cvtag{Computer Vision}
% \cvtag{Gym}
% \cvtag{Reading}
% \cvtag{Toronto Raptors}
% \cvtag{Basketball}
% \cvtag{Image Processing}
